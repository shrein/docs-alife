%\chapter{Introduction} \label{ch:intro}
\section{Goals}
Dynamical bipedal walking has been a key objective in robotics since
its origins, due to the human curiosity about artificial
anthropomorphic beings which gave rise to the robot concept itself
\cite{Capek3R.U.R.}. As could be seen in most industrialized
countries, the industrial manipulators have found a wide adoption, and
there is little space to a major boost in the area
\cite{Yonemoto85TECHNOLOGY}. Although, the interest is shifting from
an industrial point of view to a more domestic one
\cite{Asami94Robots}, where robots can be seen as additional aids to
human daily tasks.

But, in order to accompany humans, the robot must be able to fluidly
move through all the environments in which the human can, and those
environments are  devised to adapt well to anthropomorphic beings:
factories, vehicles, houses, sidewalks, and shopping malls, among
others. This way, a robot made to perform well in arbitrary
environments will have a great advantage if it is anthropomorphic, so
that it could serve well as an personal assistant
\cite{Dario01Humanoids}.


The interest on biped robotics is not only for biorobotics
itself. Another reason to research anthropomorphic motion is the
understanding of human morphology, mechanics and control, from a
medical point of view, where robotics could serve as a testing
scenario to both theories and technologies concerning human motion
(for an example see \cite{Woo06Biomechanics}) and, probably, provide
technological aids and substitutes to body parts when an impairment is
present \cite{Hermini01Proposal}. 


Another motivation to the research of biped walking is related to the
fact that anthropomorphic motion planning and control is a complex
problem that includes nonlinear and non-holonomic systems
\cite{Basdogan96Nonlinear}, complex computing tasks, adaptability to
unknown and unstructured environments \cite{Cheng00Dynamic}, among
others, and is useful to test different mechanical, electronic,
computing and control techniques applicable to diverse areas. As it is
remarked by Craig \cite{Craig89Introduction}, it should not be forgot
that the predominant dynamics algorithm for open-chain mechanisms was
developed \cite{Stepanenko6Dynamics} and refined \cite{Orin9Kinematic}
while working in biped walking problems.


Among the different problems faced in anthropomorphic motion, biped
walking is one of the most difficult because its intrinsic
instability. In contradistinction to wheeled mobile robotics and
stable legged robotics, biped robotics must allow locally instable
motion in order to attain a fluid locomotion. Additionally, there are
several problems in biped locomotion other than stability. A bipedal
walker must be in capacity of choose the best path to reach an
objective, avoid obstacles, tolerate high perturbations, perform well
in unstructured environments, and move with an optimal energy
consumption.


This way, while already a problem broadly studied, the problem of
biped walking is still open for several key goals not currently
attained, such as optimal energy consumption, objective-based planning
and walking in less structured environments. However, there are
recent, and not so recent, notable contributions in this field given
by diverse methods: active control, passive dynamic walking, and
computational intelligence; which are giving a stronger basis for
major advances in the field, fueled by a strong academic and
commercial motivation.

Nonetheless, here we study a narrower problem, in which biped robotics
combines with dynamical neural neurons and evolutionary robotics to
pursue biped walking control by emergence, i.e. without the explicit
specification those parameters required in order to achieve the
desired behavior with the control scheme chosen.

In a intuitive way, the problem studied can be broadly described as to
provide a biped robot with an optimal locomotion. Nonetheless, since
the optimality referred can be thought as a performance measure, it is
convenient to be more specific in its definition. Based on the
preliminaries in previous sections, the relevant goals to be
accomplished by biped locomotion proposed are:

\begin{itemize}

\item To conserve static stability: The ability to conserve
  equilibrium while in static state or in quasi-static motion.

\item To walk conserving dynamic stability: The ability to walk, or
  locomote without leaving the contact to the floor, at velocities
  high enough to make considerable the inertial forces maintaining
  stability. That means the robot should walk indefinitely without
  falling if there are no obstacle in its way.

\item To locally minimize energy consumption: For a given path
  minimize the energy consumption of the gait pattern.

\item To react to external perturbations: Sense and compensate
  external perturbations that affect locomotion, including external
  forces and changes in environment parameters.

\item To plan gait trajectories to attain specific objectives: The
  ability to choose a suitable trajectory to reach a desired state,
  not only in terms of path, but also joint coordination.

\item To move across unstructured environments: Perform biped
  locomotion in environment with variable conditions which are not
  previously given to the robot, and plan according to perception of
  those environment variable conditions.
\item To globally minimize energy consumption: For a given objective
  in an arbitrary environment, select and perform the gait trajectory
  that minimizes energy consumption.

\end{itemize}


Although non strictly, the order of goals presented has in mind a
growing complexity required to satisfy them. That implies that in some
cases it would be useful to satisfy, at least partially, a previous
one before the one following it.

The last goal proposed, the global minimization of energy consumption,
has not been satisfied by any of the works reviewed so far, and the
goals of motion across unstructured environments and planning of
trajectories to attain specific objectives have been taken into
account by few of them . Particularly, in the area of computational
intelligence it is necessary first to address consistently the problem
of planning previous to studying the remaining goals.


Therefore, the specific problem defined is to perform the motion
planning and control of biped locomotion in a computer-simulated
environment using a goal-oriented integrated architecture based on
computational intelligence. The biped model defined is a rigid body
linked model, with the lesser number of joints required to move in a
two-dimensional environment.


The motion planning is understood as the process to detail a motion
task into a sequence of reference actions, while the motion control is
the process that transforms a sequence of reference actions into
control signals applied to robot actuators (applying torque
accordingly to each joint).The computer-simulated environment is
essentially a physics engine in which the dynamical experiments can be
performed, given a model of the system, environmental conditions and a
set of inputs.


% \section{Research Objectives}

% \subsection{General Research Objective}
% To perform the motion planning and control of biped locomotion in a computer-simulated environment using a goal-oriented integrated architecture based on computational intelligence.

% \subsection{Specific Research Objectives}

% \begin{itemize}
% \item To develop or configure a computer-simulated environment that allows to perform dynamical experiments of motion tasks with the biped
% \item To develop or configure a visualization module that allows to evaluate the performance of the simulations made in the simulation environment
% \item To design and implement algorithmically a goal-based integrated architecture of motion planning and control of biped locomotion and develop the interface with the simulation environment
% \end{itemize}


\section{Main Contributions}
The goal-oriented integrated architecture proposed is the main
contribution of the thesis proposal, since in the work reviewed there
are no goal-oriented motion planning strategies applied to biped
locomotion, and therefore there are no planning and control
integration proposals. In order to be able to move consistently across
an unstructured environment and to be useful for any specific task it
is required that the biped can pursue specific goals. The integration
of planning and control has two purposes. The first one is to allow a
uniform and hierarchical implementation across the architecture,
allowing the same adaptation strategies and analysis tools to be
applied to both planning and control systems, and performing the
planning task dynamically in the same time scale in which the control
system is working. The second one is to be able to change softly from
one operation mode to another when goals have changed, controlling
undesired high frequency components of the reference control signal.


The proposed problem will face dynamic biped walking in structured
environments without perturbations. With the contribution in
goal-oriented planning and its integration to motion control using a
computational intelligence approach, it is expected that there will be
a framework such that the motion across unstructured environments and
the global energy minimization fall in the field of study in short
time.  


\section{Contents Overview}




