%\chapter{Conclusion} \label{ch:concl}
\section{Main Contributions}
The goal-oriented control architecture using neural fields proposed
here, is the main contribution of this work. It shows that neural
fields may be used as a useful component to build motion controllers,
taking into account that they are also biologically plausible.

An additional but important contribution, related to the goal of
minimization of global energy consumption, is the optimization by
means of evolutionary algorithms and neural fields of the previous
work on pseudo-Passive Dynamic Walking controllers (see
\cite{Wisse05How}) for the Simplest Biped Walking model (as proposed
in \cite{Garcia98simplest}). With this contribution, this work
advances towards what could be the most valuable goal on biped walking
control.

\subsection{Neural Field Controller for the Inverted Pendulum}
In chapter \ref{ch:chp2}, which was published in two parts in the
Proceedings of IJCNN 2009 and GECCO 2009 (\cite{Figueredo09Applying}
and \cite{Figueredo09Evolved}), it was developed a neural field
controller that is able to solve the stability problem for the
inverted pendulum (cart-and-pole). Both a neural field controller
manually tuned, and a neural field controller parametrized using
evolutionary algorithms were presented. Also, two additional
controllers were presented for comparison: a controller using evolved
recurrent neural networks, and the non-evolved neural field controller
omitting the processing layer for output.

It could be seen that, while the recurrent neural network controller
is expressive enough to solve the problem at hand, the number of
parameters to configure is of a quadratic order in relation to the
number of neurons.

On the other hand, the neural field controller was found to be more
complex and its simulation more costly, but had some notable
advantages: The stability of its natural dynamics, and its suitability
to the problem at hand, being able to solve it with a acceptable
degree of performance for low and mid perturbations, even without
evolution.

The evolved controller performed better than the other two, is more
general because eliminates the need for manual conscious
parametrization and allows the designer to specify more precisely the
performance measure desired. Also it allows interpretation of field
potentials. and its geometrical representation allows for a small
switch between a state-space like controller to a neural field
controller. The interpretation or understanding of recurrent neural
networks tends to be difficult and even more with increasing states
involved. This makes the neural field controller not as black-box as
the a recurrent neural network controller.

\subsection{Neural Field Controller for SBW}

While the cart-and-pole biped walking model is useful as a first
approach to the problem of locomotion, in order to develop a more
realistic controller of biped walking, a model closer to human biped
walking was explored in chapter \ref{ch:chp3}, and also some new
controllers were presented.

An extended neural field control and planning architecture was
developed and applied to the stability problem for the Simplest Biped
Walking model. Also it was shown how the control architecture may be
used at a planning level by changing the control policy to be applied,
and also how it is able to minimize the global energy consumption. The
neural field control architecture was compared to the linear
controller proposed by Wisse et al. \cite{Wisse05How}, and to the
optimized but not biologically inspired Sliding-mode controller
proposed in this work.

Those controllers proposed in chapter \ref{ch:chp3}, while being
active, approach more closely Passive Dynamic Walking (PDW) than
previous works, by diminishing the cumulative control action. One of
the two controllers proposed was implemented using a control
architecture based on neural fields, which extends and formalizes the
structure of the controller for the inverted pendulum shown in the
previous chapter. Also, a parametrization using evolutionary
algorithms (HAEA by Gomez, see \cite{Gomez04Self}) was performed on
the mappings between the representation layer and the action layer of
the neural field controller proposed (also used directly in the other
controller presented).

It is important to note that the control strategies presented in
chapter \ref{ch:chp3}, would also work, mostly unaltered, for any
other system were there is a controller for which exists a parameter
so that the attraction basin widens as that parameter is increased.

While the State feedback (SF) controller of Wisse et al. keeps
constant the $k_{\phi}=100$ value in order to attain stability, both
the optimized Sliding-mode controller (OSM) and Sliding-mode-like
neural field (SMNF) controller proposed were shown to recalculate the
$k_{\phi}$ (at least) on each step using the current Poincar� section
and the embedded mapping $k_{\phi}=M(v_n^+)$, directly for the OSM,
and indirectly for the SMNF.

It was shown that the output values of the OSM take the form of a
Sliding-mode controller in the following sense:
\begin{itemize}
\item A subset of Poincar� section at heel-strike defines the section
  of state-space where the system ``slides'' along.
\item At each heel-strike the parameter $k_{\phi}$ is re-evaluated,
  effectively changing the structure of the controller in a
  discontinuous fashion.
\item The matrix of control values $M(v_n^+)$ is built so that the
  system always stays in the subset of the Poincar� section, and
  furthermore, approaches the sliding surface in finite time.
\item Once reaches the sliding surface (fixed point in the Poincar�
  section), the system stays there.
\end{itemize}

Also, it was shown that given the output function the SMNF uses, and
the dynamic estimation of the system state implemented in the
representation layer, could cause the loss of stability, as it may lag
or otherwise underestimate the required $k_{\phi}$. Nonetheless, the
SMNF approaches quite closely to the behavior in the long term of the
OSM controller, but with a soft and continuous change in the control
parameter $k_{\phi}$.

The overall performance of the SMNF controller was better than the SF
in terms of energy consumption and actuator strain, progressively
diminishing its $k_{\phi}$ value and thus the cumulative control
action. The SF controller converges faster to its fixed point, but
even in steady-state the control action stays almost unaltered. On the
other hand, the OSM controller has an slightly higher cumulative
control action than the SMNF controller, but it provides a monotonic
decrease in $k_{\phi}$ across time, behaving more properly as an
Sliding-mode controller.

The Poincar� Sections of the biped using both the SMNF controller and
the OSM controller did show how the system moves from a configuration
that requires higher $k_{\phi}$ values to configurations that require
lower ones, and that fact is exploited by the controllers.

The time simulation did show how the SMNF changes the control policy
(as parametrized by $k_{\phi}$) softly enough to provide a
qualitatively natural gait for the biped. It should be noted that
sudden changes of behavior are common in Sliding-mode controllers, but
that is mitigated in this case by three facts: 1) The neural field
applies an interpolation using the centroid of activation to calculate
the output $k_{\phi}$. 2) The neural field has natural dynamics
qualitatively equal to a low-band filter. 3) The change in the
$k_{\phi}$ occur at a non-linear point in the biped dynamics that
anyway would cause a jump in its state.

\section{Future Work}
More complex control problems could benefit from the strategy of
parametrization of neural field controllers by evolution here
developed. This way, it is left for future work the task of
integration of multiple goals in different populations and the design
of an arbitration scheme compatible with the architecture.

Also, the neural field controller presented in chapter \ref{ch:chp3}
may be extended using reinforcement learning, specially a form of
Q-learning, to efficiently identify the optimal action-values (mapping
between reinforcement and action layers). It could be done using
biologically plausible techniques, as shown by Str�sslin et al. in
\cite{Stroesslin03Reinforcement}.

Finally, further application of the extended neural field controller
architecture, used at a policy modification level, could be explored
to further asses its performance in a wider variety of problems.

