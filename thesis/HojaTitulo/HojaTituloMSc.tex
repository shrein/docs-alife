%\newpage
%\setcounter{page}{1}
\begin{center}
%\begin{figure}
%\centering%
\includegraphics[scale=0.9]{EscudoUN}\\
%\epsfig{file=HojaTitulo/EscudoUN.eps,scale=1}%
%\end{figure}
\thispagestyle{empty} \vspace*{2.0cm} \textbf{\huge
Goal-Based Control and
  Planning in Biped Locomotion Using Computational Intelligence
  Methods}\\[5.0cm]
\Large\textbf{Juan Jos� Figueredo Uribe}\\[6.0cm]
\small Universidad Nacional de Colombia\\
Facultad de Ingenier�a, Departamento de Ingenier�a de Sistemas e Industrial\\
Bogot� D.C., Colombia\\
2012\\
\end{center}

\newpage{\pagestyle{empty}\cleardoublepage}

\newpage
\begin{center}
  \thispagestyle{empty} \vspace*{0cm} \textbf{\huge Goal-Based Control
    and Planning in Biped Locomotion Using Computational Intelligence
    Methods}\\[2.5cm]
  \Large\textbf{Juan Jos� Figueredo Uribe}\\[2.0cm]
  \small Thesis submitted as partial fulfillment of the requirements for the degree of:\\
  \textbf{Magister en Ingenier�a de Sistemas y Computaci�n}\\[2.0cm]
  Tutor:\\
  Jonatan G�mez Perdomo, PhD\\[2.0cm]
  Research Area:\\
  Computational Intelligence\\
  Research Group:\\
  Grupo de Investigaci�n en Vida Artificial - Alife\\[2.5cm]
  Universidad Nacional de Colombia\\
  Facultad de Ingenier�a, Departamento de Ingenier�a de Sistemas e Industrial\\
  Bogot� D.C., Colombia\\
  2012\\
\end{center}

\newpage{\pagestyle{empty}\cleardoublepage}

\newpage \textbf{\LARGE Resumen}
\addcontentsline{toc}{chapter}{\numberline{}Resumen}\\\\
Este trabajo explora la aplicaci�n de campos neuronales, a tareas de
control din�mico en el domino de caminata b�peda.  

En una primera aproximaci�n, se propone una arquitectura de control
que usa campos neuronales en 1D. Esta arquitectura de control es
evaluada en el problema de estabilidad para el p�ndulo invertido de
carro y barra, usado como modelo simplificado de caminata b�peda. El
controlador por campos neuronales, parametrizado tanto manualmente
como usando un algoritmo evolutivo (EA), se compara con una
arquitectura de control basada en redes neuronales recurrentes (RNN),
tambi�n parametrizada por por un EA. El controlador por campos
neuronales parametrizado por EA se desempe�a mejor que el
parametrizado manualmente, y es capaz de recuperarse r�pidamente de
las condiciones iniciales m�s problem�ticas.

Luego, se desarrolla una arquitectura extendida de control y
planificaci�n usando campos neurales en 2D, y se aplica al problema
caminata bipeda simple (SBW). Para ello se usa un conjunto de valores
�ptimos para el par�metro de control, encontrado previamente usando
algoritmos evolutivos. El controlador �ptimo por campos neuronales
obtenido se compara con el controlador lineal propuesto por Wisse et
al., y a un controlador �ptimo tabular que usa los mismos par�metros
�ptimos. Si bien los controladores propuestos para el problema SBW
implementan una estrategia activa de control, se aproximan de manera
m�s cercana a la caminata din�mica pasiva (PDW) que trabajos previos,
disminuyendo la acci�n de control acumulada.
\\

\textbf{campos neuronales, neurocontrol, rob�tica evolutiva,
  vida artifical, control �ptimo, caminata b�peda, p�ndulo invertido, caminata b�peda pasiva}.\\

\newpage
\textbf{\LARGE Abstract}\\\\
This work explores the application of neural fields to dynamical
control tasks in the domain of biped walking.

In a first approximation, a controller architecture that uses 1D
neural fields is proposed. This controller architecture is evaluated
using the stability problem for the cart-and-pole inverted pendulum,
as a simplified biped walking model.  The neural field controller is
compared, parameterized both manually and using an evolutionary
algorithm (EA), to a controller architecture based on a recurrent
neural neuron (RNN), also parametrized by an EA. The non-evolved
neural field controller performs better than the RNN controller. Also,
the evolved neural field controller performs better than the
non-evolved one and is able to recover fast from worst-case initial
conditions.

Then, an extended control and planning architecture using 2D neural
fields is developed and applied to the SBW problem. A set of optimal
parameter values, previously found using an EA, is used as parameters
for neural field controller. The optimal neural field controller is
compared to the linear controller proposed by Wisse et al., and to a
table-lookup controller using the same optimal parameters. While being
an active control strategy, the controllers proposed here for the SBW
problem approach more closely Passive Dynamic Walking (PDW) than
previous works, by diminishing the cumulative control action.
\\[2.0cm]

\textbf{neural fields, neurocontrol, evolutionary robotics,
  artificial life, optimal control, biped walking, inverted pendulum, passive dynamic walking}\\
