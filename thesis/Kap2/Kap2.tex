\chapter{Cap\'{\i}tulo 1}
Los cap\'{\i}tulos son las principales divisiones del documento. En estos, se desarrolla el tema del documento. Cada cap\'{\i}tulo debe corresponder a uno de los temas o aspectos tratados en el documento y por tanto debe llevar un t\'{\i}tulo que indique el contenido del cap\'{\i}tulo.\\

Los t\'{\i}tulos de los cap\'{\i}tulos deben ser concertados entre el alumno y el director de la tesis  o trabajo de investigaci\'{o}n, teniendo en cuenta los lineamientos que cada unidad acad\'{e}mica brinda. As\'{\i} por ejemplo, en algunas facultades se especifica que cada cap\'{\i}tulo debe corresponder a un art\'{\i}culo cient\'{\i}fico, de tal manera que se pueda publicar posteriormente en una revista.\\

\section{Subt\'{\i}tulos nivel 2}
Toda divisi\'{o}n o cap\'{\i}tulo, a su vez, puede subdividirse en otros niveles y s\'{o}lo se enumera hasta el tercer nivel. Los t\'{\i}tulos de segundo nivel se escriben con min\'{u}scula al margen izquierdo y sin punto final, est\'{a}n separados del texto o contenido por un interlineado posterior de 10 puntos y anterior de 20 puntos (tal y como se presenta en la plantilla).\\

\subsection{Subt\'{\i}tulos nivel 3}
De la cuarta subdivisi\'{o}n en adelante, cada nueva divisi\'{o}n o \'{\i}tem puede ser se\~{n}alada con vi\~{n}etas, conservando el mismo estilo de \'{e}sta, a lo largo de todo el documento.\\

Las subdivisiones, las vi\~{n}etas y sus textos acompa\~{n}antes deben presentarse sin sangr\'{\i}a y justificados.\\

\begin{itemize}
\item En caso que sea necesario utilizar vi\~{n}etas, use este formato (vi\~{n}etas cuadradas).
\end{itemize} 