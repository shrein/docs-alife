\documentclass[12pt,fleqn,openany,letterpaper,pagesize]{scrbook}

\usepackage[latin1]{inputenc}
\usepackage[numbers]{natbib}
\usepackage[spanish,english]{babel}
\usepackage[left=4cm,top=3cm,right=3cm,bottom=3cm]{geometry}
\usepackage{fancyhdr}
%\usepackage{epsfig}
\usepackage[pdftex]{graphicx}
%\usepackage{epic}
%\usepackage{eepic}
\usepackage{algorithm}
\usepackage{algorithmic}
\usepackage{listings}
\usepackage{amsfonts}
\usepackage{amssymb}
\usepackage{amsmath}
\usepackage{mparhack}
%\usepackage{phdthesis}
%\usepackage{titlepage_nat_uni}
\usepackage{subfig}
%\usepackage{ifpdf}
\usepackage[section]{placeins}

\graphicspath{{./images/}}

%\usepackage{threeparttable}
%\usepackage{amscd}
%\usepackage{here}
%\usepackage{lscape}
%\usepackage{tabularx}
%\usepackage{subfigure}
%\usepackage{longtable}


%\usepackage{rotating} %Para rotar texto, objetos y tablas seite. No se ve en DVI solo en PS. Seite 328 Hundebuch
                        %se usa junto con \rotate, \sidewidestable ....

\graphicspath{{./images/}}

\renewcommand{\theequation}{\thechapter-\arabic{equation}}
\renewcommand{\thefigure}{\textbf{\thechapter-\arabic{figure}}}
\renewcommand{\thetable}{\textbf{\thechapter-\arabic{table}}}


\pagestyle{fancyplain}%\addtolength{\headwidth}{\marginparwidth}
%\textheight22.5cm \topmargin0cm \textwidth16.5cm
%\oddsidemargin0.5cm \evensidemargin-0.5cm%
\renewcommand{\chaptermark}[1]{\markboth{\thechapter\; #1}{}}
\renewcommand{\sectionmark}[1]{\markright{\thesection\; #1}}
\lhead[\fancyplain{}{\thepage}]{\fancyplain{}{\rightmark}}
\rhead[\fancyplain{}{\leftmark}]{\fancyplain{}{\thepage}}
\fancyfoot{}
\thispagestyle{fancy}%


\addtolength{\headwidth}{0cm}
\unitlength1mm %Define la unidad LE para Figuras
\mathindent0cm %Define la distancia de las formulas al texto,  fleqn las descentra
\marginparwidth0cm
\parindent0.5cm %Define la distancia de la primera linea de un parrafo a la margen

%Para tablas,  redefine el backschlash en tablas donde se define la posici\'{o}n del texto en las
%casillas (con \centering \raggedright o \raggedleft)
\newcommand{\PreserveBackslash}[1]{\let\temp=\\#1\let\\=\temp}
\let\PBS=\PreserveBackslash

%Espacio entre lineas
\renewcommand{\baselinestretch}{1.1}

%Neuer Befehl f\"{u}r die Tabelle Eigenschaften der Aktivkohlen
\newcommand{\arr}[1]{\raisebox{1.5ex}[0cm][0cm]{#1}}

%Neue Kommandos
%\usepackage{Befehle}
